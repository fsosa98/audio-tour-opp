\chapter{Opis projektnog zadatka}
		
		\textit{
			Cilj ovog projekta je stvoriti funkcionalan sustav koji bi modernizirao muzeje. Korisnici koji posjećuju muzej sada to mogu činiti na zanimljiv način uz audio vodič,  a oni koji nisu u mogućnosti doći do samog muzeja, dobivaju priliku doživjeti eksponate preko slika i zvučnih zapisa koje nudi ova web aplikacija.  Ovakav sustav donosi koristi i za korisnike i za vlasnike muzeja koji sada mogu lakše privući i informirati svoje posjetitelje. 
			Sustav je osmišljen kao web aplikacija s četiri kartice. Prva kartica je početna stranica koja sadrži ime muzeja kao naslov, u gornjem lijevom kutu podatke trenutno prijavljenog korisnika (korisničko ime) te u gornjem desnom kutu gumb „Login“ koji vodi na karticu za prijavu u sustav. Ispod naslova se nalazi podnaslov koji nosi ime prve grupe objekata, u ovom slučaju „Rim“. Ispod tog naslova su poredane četiri slike – dvije po redu – koje klikom na pojedinu sliku odvode do kartice sa sadržajem relevantnim za taj određeni eksponat. 
			Kartica određenog objekta (eksponata) sadrži na sredini naziv eksponata ispod kojeg se nalazi fotografija istog velikih dimenzija. Ispod fotografije je tekst koji opisuje taj objekt, a na dnu teksta možemo pronaći „tag“ u obliku grupa koji govori kojoj grupi taj objekt pripada. Nakon teksta dolazi gumb za pokretanje audio vodiča koji je obavezno kraći od 3 minute. U gornjem lijevom kutu se nalazi gumb „Početna“ koji nas vraća na početnu stranicu. Na samom dnu stranice dolazimo do QR koda za taj objekt koji se može skenirati pomoću pametnog telefona, a jednom kad je skeniran, reproducira se audio vodič za taj objekt. 
			Klikom na gumb „Login“ (koji se nalazi na početnoj stranici), na ekranu se pojavljuje opcija prijave u sustav. Od korisnika se traži da unese svoje korisničko ime i lozinku u dva odvojena prozora.  Ispod toga se nalazi tekst „Ako niste registrirani, kliknite ovdje“ te se klikom na link otvara kartica za registraciju novih korisnika. U gornjem lijevom kutu se nalazi gumb „Početna“ koji vodi natrag na početnu stranicu aplikacije. Jednom kad je korisnik prijavljen u sustav, u gornjem desnom kutu početne stranice može vidjeti gumb „Odjava“ kojim se odjavljuje iz sustava.
			Ako se korisnik odluči registrirati u sustav, doći će do stranice namijenjene za registraciju koja također u gornjem lijevom kutu ima gumb „Početna“ iste funkcionalnosti kao i ranije. Prilikom registracije, od korisnika se traži da unese redom ime, prezime, e-mail adresu, korisničko ime i lozinku. Jednom kad korisnik pritisne gumb „Podnesi“, na navedenu e-mail adresu mu se šalje potvrda o registraciji koja u sebi sadrži link na početnu stranicu. Pritiskom na link, korisnik esencijalno potvrđuje svoju registraciju i postaje registrirani korisnik ili administrator – ovisno o prethodnoj definiciji od strane vlasnika sustava. Nakon uspješne registracije u sustav, korisnik na svoj navedeni e-mail prima poruku koja sadrži njegove pristupne podatke i dobrodošlicu. 
			Sustav podržava četiri različita aktera – vlasnika sustava, administratore, registrirane korisnike i neregistrirane korisnike. Svaka od ovih uloga donosi različite razine ovlasti i mogućnosti te je doživljaj web aplikacije drugačiji za svakog od ova četiri aktera. 
			Vlasnik sustava može biti samo jedan te on ima najvišu razinu ovlasti. Vlasnik je u mogućnosti definirati maksimalno pet administratora i njihova prava te im dati pristupne podatke. 
			Jednom kad administrator dobije svoje pristupne podatke od vlasnika sustava, može upisati podatke o sebi – ime, prezime i e-mail adresu. Administrator je taj koji uređuje ovu web aplikaciju, odnosno dodaje sadržaj u obliku teksta. Njemu se na početnoj stranici nudi mogućnost dodavanja novih grupa objekata kao i uređivanje i dodavanje novih objekata unutar postojećih grupa. 
			Dodatno, vlasnik sustava i svi administratori imaju pristup određenim statistikama i podacima kao što su: }
		
		\begin{packed_item}
			\item \textit{Ukupna posjećenost stranice}
			\item \textit{Prikaz broja pregleda stranica koji može biti sortiran po ukupnom broju prikaza ili vremenu zadržavanja na stranici}
			\item \textit{Broj reprodukcija pojedinog audio vodiča}
			\item \textit{Broj korisnika koji su pokrenuli audio vodič na pametnom telefonu }
			\item \textit{	Broj i korisnička imena trenutno aktivnih registriranih korisnika i drugih administratora}
	
		\end{packed_item}
		
		\textit{	Registrirani i neregistrirani korisnik ima pristup sadržaju na početnoj stranici te svim ponuđenim podacima o objektima u muzeju – opis, slika, audio vodič i QR kod. Razlikuju se u tome što registrirani korisnici imaju pristup i promo materijalima turističke zajednica i muzeja, dok neregistrirani korisnici nemaju tu opciju. }
		\eject
			


		
		
		
	