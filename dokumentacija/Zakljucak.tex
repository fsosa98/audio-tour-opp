\chapter{Zaključak i budući rad}
		
		 \textit{Cilj našeg projekta iz kolegija Oblikovanje programske potpore, bio je izraditi web stranicu koja bi poboljšala ukupno iskustvo odlaska u muzej. Kroz izradu ovog projekta, naučili smo raditi u timu i dobili smo priliku stvoriti nešto korisno i konkretno što je primjenjivo u stvarnom životu. Rad na projektu bio je podijeljen u dvije faze. 
                U prvoj fazi smo napravili osnove i okosnicu krajnjeg produkta. Za početak smo se upoznali s mentorom, ali i međusobno, kroz nekoliko sastanaka i komunikaciju preko Slacka, razjasnili smo si što sve treba napraviti u ovom ciklusu i dogovorili smo se kako će naša web stranica otprilike izgledati. Podijelili smo rad na sve članove tima – neki su radili frontend, neki backend, a svi smo zajedno doprijenili izradi dokumentacije. Kroz tu prvu fazu smo definirali funkcionalne i nefunkcionalne zahtjeve, ostale važne zahtjeve, tehnologije koje ćemo koristiti te općeniti plan rada. Za potrebe dokumentacije, naučili smo se služiti alatom Astah u kojem smo izradili nekolicinu sekvencijskih dijagrama, dijagrama obrazaca uporabe i dijagrama razreda. Pokazalo se kako su ti dijagrami od izrazite važnosti za projekt jer jasno pokazuju veze između različitih sudionika i dijelova sustava što nam je pomoglo razumjeti procese koje moramo pokriti pri izradi ove web stranice. To znanje će nam zasigurno koristiti i pri izradi bilo kojeg drugog budućeg projekta. Do kraja prve faze smo uspostavili bazu podataka te izradili osnovne dijelove stranice. Imali smo početnu stranicu, stranicu za prijavu i registraciju u sustav te stranice objekata. 
                U drugoj fazi smo ušli u detalje ovog projektnog zadatka. Nastavili smo s redovnim sastancima na kojima smo složno definirali rad iz tjedna u tjedan. Dodali smo sve ostale funkcionalnosti sustava za sve aktore što je detaljno opisano na početku ovog dokumenta pa ćemo ovdje navesti samo nekoliko primjera:  uređivanje objekata i grupa za administratore, pregled promo materijala i objekata za registrirane korisnike te pregled statistike za vlasnika i administratore. Dodali smo zvučne zapise za svaki od objekata i omogućili skeniranje QR kodova koji vode do tih audio zapisa za registrirane i neregistrirane korisnike. Dovršili smo dokumentaciju nadopunivši je opisom korištenih tehnologija i alata, novim vrstama dijagrama, uputama za puštanje u pogon i ovim zaključkom. Na kraju smo sve istestirali po pravilima kolegija što je također dokumentirano ranije u ovom pdf-u. 
                U budućnosti ne očekujemo neke veće poteškoće s obzirom na to da svaki dio sustava funkcionira. Naravno, kako se sustav može primjeniti na nebrojeno mnogo muzeja, moguće je prilagoditi ga svakom po potrebi. Uz redovno održavanje i praćenje sustava, ne bi trebalo biti nekih značajnih problema u radu same stranice. 
                Smatramo da je ovaj projekt bio izuzetno korisno iskustvo koje je nam je dalo ideju kako izgleda izrada zajedničkog projekta u stvarnom svijetu. Upoznali smo neke nove tehnologije, naučili smo pisati dokumentaciju i dogovarati se i raditi u timu. 
}
		
		 
		
		\eject 